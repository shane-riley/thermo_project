%%% use 10pt options with the asme2ej format
\documentclass[10pt,cleanfoot]{asme2ej}

\usepackage{graphicx} %% for loading jpg figures
\usepackage{amsmath}

%% The class has several options
%  onecolumn/twocolumn - format for one or two columns per page
%  10pt/11pt/12pt - use 10, 11, or 12 point font
%  oneside/twoside - format for oneside/twosided printing
%  final/draft - format for final/draft copy
%  cleanfoot - take out copyright info in footer leave page number
%  cleanhead - take out the conference banner on the title page
%  titlepage/notitlepage - put in titlepage or leave out titlepage
%  
%% The default is oneside, onecolumn, 10pt, final

\title{Optimization of a Regenerative \\
Reheat Rankine Cycle with an Organic Rankine Cycle}

%%% first author
\author{Shane Riley
    \affiliation{
	Undergraduate, Mechanical Engineering\\
	Swanson School of Engineering\\
	Department of Mechanical Engineering and Materials Science\\
	University of Pittsburgh, Pittsburgh, PA 15261, USA\\
    Email: shane.riley@pitt.edu
    }	
}

%%% second author
%%% remove the following entry for single author papers
%%% add more entries for additional authors
\author{Gabriel Vogel
    \affiliation{
	Undergraduate, Mechanical Engineering\\
	Swanson School of Engineering\\
	Department of Mechanical Engineering and Materials Science\\
	University of Pittsburgh, Pittsburgh, PA 15261, USA\\
        Email: gtv2@pitt.edu
    }
}

\begin{document}

\maketitle    

%%%%%%%%%%%%%%%%%%%%%%%%%%%%%%%%%%%%%%%%%%%%%%%%%%%%%%%%%%%%%%%%%%%%%%
\begin{abstract}
{\it 
The Rankine Cycle is fundamental to the operation of utility-scale power plants. While the simplest configuration of the cycle employs only four devices (pump, turbine, boiler, condenser), such a cycle leaves additional efficiency on the table. 
Greater efficiency can be achieved by adding reheat processes, feedwater heaters, and an Organic Rankine Cycle with a lower boiling point working fluid. These additions increase the cost and complexity of the cycle, but also increase the efficiency. At the utility-scale, the latter is vastly more important. To analyze and optimize a system with a closed feedwater heater (C.F.H.), a open feedwater heater (O.F.H.), a reheat process, and an Organic Rankine Cycle (ORC), a script is presented based upon thermodynamic laws, project constraints, and guiding assumptions. This script is run using the Engineering Equation Solver (EES) and a number of input parameters, and outputs a measure of overall thermal efficiency. Using the model as a guide, an optimal set of input parameters is reached for maximum thermal efficiency.
}
\end{abstract}

%%%%%%%%%%%%%%%%%%%%%%%%%%%%%%%%%%%%%%%%%%%%%%%%%%%%%%%%%%%%%%%%%%%%%%
\begin{nomenclature}
\entry{$P$}{Pressure of fluid [kPa].}
\entry{$T$}{Temperature of fluid [C].}
\entry{$x$}{Quality of fluid [-]. When superheated or subcooled, the quality is defined as 100 [-] or -100 [-], respectively, as per the EES standard.}
\entry{$s$}{Real specific entropy of fluid  [kJ/kg-K].}
\entry{$s_{ideal}$}{Ideal specific entropy of fluid  [kJ/kg-K].}
\entry{$h$}{Real specific enthalpy of fluid [kJ/kg].}
\entry{$h_{ideal}$}{Ideal specific enthalpy of fluid [kJ/kg].}
\entry{$m_{main}$}{Mass flow of main line in steam system [kg/s].}
\entry{$m_{hex}$}{Mass flow of ORC [kg/s].}
\entry{$y$}{Mass flow fraction released to the C.F.H. [-].}
\entry{$z$}{Mass flow fraction released to the O.F.H. [-].}
\entry{$\eta_{th}$}{Overall thermal efficiency [-].}
\entry{$\eta_{pump}$}{Isentropic efficiency of pumps [-].}
\entry{$\eta_{turbine}$}{Isentropic efficiency of turbines [-].}
\entry{$Q$}{Thermal power transferred through a device [kW].}
\entry{$W$}{Work power transferred through a device [kW].}
\entry{$HEX$}{Heat Exchanger.}
\entry{$C.F.H.$}{Closed Feedwater Heater.}
\entry{$O.F.H.$}{Open Feedwater Heater.}
\entry{$ORC$}{Organic Rankine Cycle.}
\end{nomenclature}

%%%%%%%%%%%%%%%%%%%%%%%%%%%%%%%%%%%%%%%%%%%%%%%%%%%%%%%%%%%%%%%%%%%%%%
\section{Introduction}

\begin{figure} 
\centerline{\includegraphics[width=4.5in]{figure/labels.jpg}}
\caption{Labelled Diagram}
\label{labels.jpg}
\end{figure}

Modern society owes itself to the study of thermodynamics, and specifically the Rankine Cycle. Standard of living depends on products made at high-scale and low-cost, and this type of production relies on cheap and accessible energy. The easiest way that humanity has found to produce energy in bulk is by using thermodynamic cycles. As such, it seems no coincidence that most innovations in thermodynamics occurred right before and during the Second Industrial Revolution [1]. The quantities of energy necessary to move humanity forward depended on an understanding of thermodynamic processes and cycles. While other forms of utility-scale power generation have evolved today that use other methods of producing electricity (photovoltaics, hydro-power, wind-power, etc.), the vast majority of power generation is still done using thermal power plants and thermodynamic cycles [1].

In order to generate work in a thermal power plant, thermodynamic cycles utilize a temperature differential created by a fuel (coal, gas, or radioactive material in a reactor). To maximize the work generated per heat in (thermal efficiency), careful consideration must be taken in the selection of the cycle. The Carnot Theorem provides guidance by predicting an absolute maximum of thermal efficiency for the temperature differential in question [1].

The Rankine Cycle, named after Scottish engineer William Rankine, uses the phase change of a working fluid to create isobaric processes during heating and cooling [1]. That, combined with roughly isentropic processes at the pump and turbine, makes the T-S diagram of the Rankine Cycle closely mimic that of the Carnot Cycle, resulting in decent thermal efficiency. While this is the case, there are ways to increase the efficiency further. The analysis presented uses three methods: An Organic Rankine Cycle, two feedwater heaters, and a reheat process. Using all of these additional methods at once vastly increases the complexity of the cycle, necessitating the use of iterative analysis in order to maximize thermal efficiency.

%%%%%%%%%%%%%%%%%%%%%%%%%%%%%%%%%%%%%%%%%%%%%%%%%%%%%%%%%%%%%%%%%%%%%%
\section{Methodology}

To perform an efficiency optimization of state variables, the Engineering Equation Solver (EES) serves as a useful tool. EES provides simple lookups for state information regarding both working fluids, and can solve a system of equations easily and out of order, as long as sufficient relations are provided. Additionally, EES allows for modulating inputs for optimization using parametric tables.

\subsection{Constraints}

As per the project description, the following constraints will be followed:

\begin{enumerate}
\item
The primary loop will use water,
\item
The secondary loop will use R-134a,
\item
Maximum turbine inlet conditions are P = 25 [MPa] and T = 600 [C],
\item
Minimum turbine quality is 90\%,
\item
Maximum reheat temperature is T = 600 [C],
\item
Pumps cannot pump any vapor (fluid must be quality of 0 [-] or subcooled),
\item
Ambient temperature is T = 25 [C],
\item
Minimum temperature differential for heat transfer devices is 15 [C],
\item
Isentropic efficiencies of pumps and turbines are 60\% and 85\%, respectively.
\end{enumerate}

\subsection{Thermodynamic Assumptions}

To simplify the analysis, the following assumptions will be made:
\begin{enumerate}
\item
Gravitational potential and kinetic energies are small relative to the internal energies of the working fluids, and can be ignored.
\item
All plumbing between devices contains fluid of constant state (i.e. no friction from fluid flow that would increase entropy).
\item
Condensers, boilers, heat exchangers and feedwater heaters are isobaric devices.
\item
The steam trap is an isenthalpic device.
\item
The C.F.H, O.F.H, and HEX are adiabatic.
\end{enumerate}

\subsection{Calculating Heats and Works}

\begin{figure} 
\centerline{\includegraphics[width=4.5in]{figure/carnot.png}}
\caption{Carnot Cycle}
\label{carnot.png}
\end{figure}

The power moving into or out of a working fluid can be measured as a difference in specific enthalpies multiplied by the mass flow rate of the fluid and the fraction of the total mass flow (dictated by y, z, and their complements). This is reasonable given the assumptions made in the analysis. Whether the power constitutes work or heat depends on the device being analyzed.

In the EES script, power direction is handled such that all values of Q and W come out as positive, given basic assumptions about which way energy should be flowing. That way, a negative sign in any work or heat quantity raises alarm. The proper signage is handled later when calculating net work and thermal efficiency. Examples are shown below:

\begin{equation}
W_{turbine,LP2} = m_{main} * (1-y-z) * (h[12] - h[13])
\label{Turbine work out}
\end{equation}

\begin{equation}
W_{pump,HP} = m_{main} * (h[6] - h[5])
\label{Pump work in}
\end{equation}

\begin{equation}
Q_{boiler,1} = m_{main} * (h[8] - h[7])
\label{Boiler heat in}
\end{equation}

\begin{equation}
Q_{out} = m_{hex} * (h[17] - h[14])
\label{Condenser heat out}
\end{equation}

\begin{equation}
Q_{hex} = m_{hex} * (h[16] - h[15])
\label{Heat across exchangers}
\end{equation}

The heat exchanger heat will be equal for each side, and the closed feedwater heater operates as a heat exchanger. The reheat portion of the boiler is defined as $Q_{boiler,2}$.

\subsection{Isobaric Devices}

For a device that is assumed isobaric, the states across the device are assigned equal pressure. An example is shown below:

\begin{equation}
P[14] = P[17]
\label{Isobaric device}
\end{equation}

For the O.F.H, there are four states with equal pressure (3 relations).

\subsection{Isenthalpic Trap}

The steam trap is isenthalpic. The relation is intuitive:

\begin{equation}
h[3] = h[4]
\label{Isobaric device}
\end{equation}

\subsection{Open Feedwater Heater}

In addition to operating at a constant pressure, the O.F.H. will experience no change in enthalpy. As such, we can relate the enthalpies entering and exiting the O.F.H. using mass flow fractions and specific enthalpies:

\begin{equation}
(z * h[12]) + ((1-y-z) * h[2]) + (y * h[4]) = h[5]
\label{O.F.H. enthalpies}
\end{equation}

\subsection{Inefficiencies of Pumps and Turbines}

Using two state variables at the inlet and pressure at the outlet, it is possible to use isentropic efficiency to find outlet state information. To do this, specific entropy is determined at the input. Then, ideal enthalpy at the outlet is determined assuming ideal entropy. Finally, the real enthalpy at the outlet is determined using the isentropic efficiency. This final relation varies between pumps and turbines:

\begin{equation}
\eta_{pump} = \frac{h[5] - h_{ideal}[6]}{h[5] - h[6]}
\label{Pump entropy}
\end{equation}

\begin{equation}
\eta_{turbine} = \frac{h[8] - h[9]}{h[8] - h_{ideal}[9]}
\label{Turbine entropy}
\end{equation}

\subsection{Known-Goods for Analysis}

With fundamental relations established, additional parameters can be set based on thermodynamic intuition:

\begin{enumerate}
\item
Heat flow across a temperature differential produces entropy that scales with the size of the differential. As such, an optimal system will use the minimum allowable temperature differences for HEX/C.F.H./condenser operation. The outlet temperatures are set with a 15 [C] difference (the condenser is set to 15 [C] above ambient).
\item
It is wasteful to cool a working fluid beyond the saturation line. As such, the qualities for all three pump inlets are set to 0 [-].
\item
Using the maximum temperature differential for the entire system is optimal (raises Carnot efficiency). As such, the first turbine inlet state is set to the maximum allowable temperature and pressure (superheat).
\item
The maximum reheat temperature is optimal for the main loop, as it drops the heat removal (HEX) pressure.
\item
In order to maximize heat recovery, the y-fraction should transfer as much heat through the C.F.H. as possible. The y-fraction quality after the C.F.H. is set to 0 [-] to achieve this.
\item
For the ORC, it is ideal to create no superheat in the refrigerant vapor. The pre-turbine refrigerant quality is set to 1 [-].
\item
The mass flow rate of the main loop will simply scale heats and works evenly, resulting in the same overall efficiency. For simplicity, the mass flow for water is set to 1 [kg/s].
\end{enumerate}

\subsection{Optimization of Remaining Parameters}

After all guiding assumptions and known-goods are placed into the model, there remain 4 degrees of freedom. These are satisfied by specifying 4 state variables as inputs.

In the final analysis, the following variables are chosen as input parameters for optimization:

\begin{enumerate}
\item
Pressure at the C.F.H. (P[9])
\item
Pressure at the reheat section of the boiler (P[10])
\item
Pressure of the O.F.H. (P[12])
\item
Pressure at HEX (water-side) (P[13])
\end{enumerate}

The first two parameters are optimized in tandem while the latter two are set to 200 and 100 [kPa], respectively. Once an ideal set of the first two parameters is reached, the O.F.H. pressure and then the HEX pressure are optimized.

\subsection{Extraneous Solutions}

While the model can produce accurate solutions for state variables, it can also create extraneous solutions with certain combinations of inputs. To remove these solutions from consideration, the following assertions are made:

\begin{enumerate}
\item
Both mass-flow fractions are positive.
\item
State entropy only decreases across devices that release heat across the control surface of the fluid (Second Law).
\item
The sum of energies into and out of the system is small compared to the heat in (should be zero, barring rounding errors).
\item
All works and heats from the EES script are positive (defined to be such by signing enthalpy differences accordingly).
\end{enumerate}

%%%%%%%%%%%%%%%%%%%%%%%%%%%%%%%%%%%%%%%%%%%%%%%%%%%%%%%%%%%%%%%%%%%%%%
\section{Results and Discussion}

By following the process of parameter optimization, the four input pressures are established and all other state variables are calculated. Using the state enthalpies, flow-rates, and flow-fractions, all heats and works are calculated. The thermal efficiency is then automatically calculated using EES:

\begin{equation}
\eta_{th} = W_{net} / Q_{in}
\label{Thermal Efficiency}
\end{equation}

The optimized state variables are shown in Table 1. State entropies and enthalpies can be calculated using the reference of choice (the EES model utilized the default option).

The methodology chosen created reasonable values for thermal efficiency and mass-flow fractions. However, the flow ratio between the ORC and the main loop was almost 15 [-]. This seems quite high--this point could serve as a starting point for further research.

Through the process of creating the model, the importance of employing assertions was quickly realized. If one of the model's relations is missing, then it creates a degree of freedom for the model that should not exist. This issue can be hidden by applying additional inputs in order to get a solution. However, if the Conservation of Mass, First Law, or Second Law is not obeyed at a device, the entire analysis is inaccurate. Therefore, the assertions established in the methodology are crucial to determining whether important relations were missing from the model.

Additionally, once the known-goods are put into the model, the efficiency starts close to the maximum. Only about a 2 percent difference in thermal efficiency exists between most favorable and unfavorable permutations of the input pressures. The presence of an analytical solution (non-iterative) for the last four inputs seems unlikely though, given the unpredictability of the maxima that appear during the parameter sweeping step.

Interestingly, the model maximizes efficiency with a near-zero pressure drop between the C.F.H. and reheat processes, meaning that the HP 2 turbine creates nearly zero work. Understanding why these pressures tend to be equal at max efficiency could be a goal for future analysis.

\begin{table}[t]
\caption{State variables}
\begin{center}
\label{table_ASME}
\begin{tabular}{c c c c }
& & & \\ % put some space after the caption
\hline
State [-] & Pressure [kPa] & Temperature [C] & Quality [-] \\
\hline
1 & 35 & 72.68 & 0 \\
2 & 340 & 72.75 & -100 \\
3 & 1500 & 198.3 & 0 \\
4 & 340 & 137.8 & 0.123 \\
5 & 340 & 137.8 & 0 \\
6 & 25000 & 144.4 & -100 \\
7 & 25000 & 183.3 & -100 \\
8 & 25000 & 600 & 100 \\
9 & 1500 & 225.6 & 100 \\
10 & 1500 & 225.6 & 100 \\
11 & 1500 & 600 & 100 \\
12 & 340 & 394.1 & 100 \\
13 & 35 & 160.5 & 100 \\
\hline
14 & 1017 & 40 & 0 \\
15 & 1592 & 40.63 & -100 \\
16 & 1592 & 57.68 & 1 \\
17 & 1017 & 40 & 0.9939 \\
\hline
\end{tabular}
\end{center}
\end{table}

\begin{table}[t]
\caption{Final model outputs}
\begin{center}
\label{table_ASME}
\begin{tabular}{c c c c}
& & & \\ % put some space after the caption
\hline
$\eta_{th} [-]$ & $y [-]$ & $z [-]$ & $\frac{m_{hex}}{m_{main}} [-]$ \\
\hline
0.423 & 0.082 & 0.078 & 14.8 \\
\hline
\end{tabular}
\end{center}
\end{table}

\begin{figure} 
\centerline{\includegraphics[width=4.5in]{figure/ts.jpg}}
\caption{Rough T-s Diagram of States}
\label{ts.jpg}
\end{figure}

%%%%%%%%%%%%%%%%%%%%%%%%%%%%%%%%%%%%%%%%%%%%%%%%%%%%%%%%%%%%%%%%%%%%%%
\section{Conclusions}

Following the methodology prescribed leads to a maximum thermal efficiency of $\eta_{th} = 0.423 [-]$, a turbine-to-C.F.H. flow fraction of $y = 0.082 [-]$, and a turbine-to-O.F.H. flow fraction of $z = 0.078 [-]$. Under these specifications, the ORC runs with 14.8 times the mass-flow of the main cycle. The following four input pressures are:

\begin{enumerate}
\item
Pressure at the C.F.H: 1500 [kPa]
\item
Pressure at the reheat: 1500 [kPa]
\item
Pressure of the O.F.H: 340 [kPa]
\item
Pressure at HEX (water-side): 35 [kPa]
\end{enumerate}

Based on the constraints, assumptions, and known-goods established in the model, efficiency is maximized at about a zero pressure-drop between the C.F.H. and the reheat process. Therefore, an optimized Rankine cycle of the assigned structure requires no second high pressure turbine. Removing the device would likely decrease the implementation/maintenance costs of the cylce and remove a point of failure. Moving forward, further consideration should likely be taken into the maximum allowable mass-flow rates, in order to determine whether the ORC flow speed predicted by the model is truly reasonable. Additionally, a higher-level analysis could be performed to weigh the predicted costs of adding additional devices against the gains of additional efficiency to find an optimal schematic for the cycle.

\begin{acknowledgment}
Special thanks goes to Tyler Zinn for his help in troubleshooting our thermodynamic model, discussing useful source material, and revealing various EES features to us.
\end{acknowledgment}

%%%%%%%%%%%%%%%%%%%%%%%%%%%%%%%%%%%%%%%%%%%%%%%%%%%%%%%%%%%%%%%%%%%%%%
% The bibliography is stored in an external database file
% in the BibTeX format (file_name.bib).  The bibliography is
% created by the following command and it will appear in this
% position in the document. You may, of course, create your
% own bibliography by using thebibliography environment as in

\begin{thebibliography}{12}
\bibitem{itemreference} {\em Rankine Cycle - Steam Turbine Cycle.}
{\em Nuclear Power.}
{Nuclear Power for Everybody.}
\end{thebibliography}

% Here's where you specify the bibliography style file.
% The full file name for the bibliography style file 
% used for an ASME paper is asmems4.bst.
\bibliographystyle{asmems4}

% Here's where you specify the bibliography database file.
% The full file name of the bibliography database for this
% article is asme2e.bib. The name for your database is up
% to you.
\bibliography{asme2e}
\end{document}
